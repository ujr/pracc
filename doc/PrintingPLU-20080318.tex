% Handout for the SWITCH Printing Workshop
% ujr/2008-02-21
% ujr/2008-03-04 last update

\magnification \magstep1
\font\titlefont=cmbx12 at 16pt
\font\sectionfont=cmbx12
\parindent=18pt
\newskip\secskip \secskip=24pt plus6pt minus 6pt
\def\section#1\par{\par\ifdim\lastskip<\secskip
  \removelastskip\penalty-200\vskip\secskip\fi
  \leftline{\sectionfont #1}%
  \nobreak\medskip\noindent}
\def\\{\hfil\break}
\def\IPP{IP\kern-.2ex P}

\newcount\footno \footno=0
\font\smallrm=cmr9 \font\smallit=cmti9 \font\smalltt=cmtt9
\def\fn#1{\unskip % guard against: ...blabla <newline> \fn{...}
  \global\advance\footno by 1% see TeXbook, p.396
  $^{\the\footno}$%
  \insert\footins{\raggedright %\parindent=1pc
  \let\rm=\smallrm \let\it=\smallit \let\tt=\smalltt \rm
  \leftskip=0pt \pretolerance=10000
  \hyphenpenalty=10000 \exhyphenpenalty=10000
  \interlinepenalty=\interfootnotelinepenalty
  \floatingpenalty=20000
  \splittopskip=\ht\strutbox \splitmaxdepth=\dp\strutbox
  \item{$^{\the\footno}$}\strut#1\strut \par\allowbreak}}

\topskip5pc
\centerline{Urs-\kern-.2ex Jakob R\"uetschi, PHZ Luzern}
\bigskip
\centerline{\titlefont Printing und Accounting}
\smallskip
\centerline{\bf an der PHZ Luzern}
\vskip2pc

\noindent
Die Informatik der PHZ Luzern betreibt rund 100 Drucker,
welche verteilt auf 9 Geb\"aude im Netzwerk 1500 Benutzern
zur Verf\"ugung stehen.
Die meisten Benutzer sind Studierende mit privaten Notebooks,
die mehrheitlich unter Windows laufen, grunds\"atzlich aber jedes
beliebige Betriebssystem mitbringen k\"onnen.
Alle Druckjobs von Studierenden werden nach dem Debit-Prinzip,
jene von Mitarbeitenden nach dem Credit-Prinzip verrechnet.


\section Die Institution

Die P\"adagogische Hochschule Zentralschweiz (PHZ) ist zust\"andig
f\"ur die Lehrerausbildung in der Zentralschweiz. Der operative
Betrieb wurde im Herbst 2003 in Luzern aufgenommen.
Die PHZ ist an den drei Standorten Luzern, Schwyz und Zug mit
unabh\"angigen Informatik-Abteilungen und -Infrastrukturen pr\"asent.

Die Informatik der PHZ Luzern ging aus der Informatik des vormaligen
Lehrerseminars (P\"adagogisches Ausbildungszentrum Musegg, kurz PZM,
heute Kurzzeitgymnasium Musegg, kurz KSM) hervor.
Das hier vorgestellte Druck- und Abrechnungssystem wurde schon
am PZM entwickelt und sp\"ater an der PHZ Luzern weiter gepflegt
und verbessert.


\section Ausgangslage und Grunds\"atze

Ein Drucksystem besteht aus f\"unf Komponenten: Clients, Server,
Drucker, dazwischen das Netzwerk und schliesslich die Jobaufbereitung
durch Druckertreiber.

Die meisten unserer Clients sind Windows Notebooks, die nicht
der Schule sondern den Studierenden und Dozierenden geh\"oren.
Daneben sind auch andere Betriebssysteme im Einsatz.
Serverseitig arbeiten wir fast ausschliesslich mit Linux-Systemen
und verwalten die User in einem LDAP Verzeichnis.
Um einer masslosen Druckwut vorzubeugen, beauftragte uns die Schulleitung,
alle Druckjobs auf Seitenbasis den Studierenden zum Zeitpunkt des Drucks
zu belasten.
Sp\"ater wurden wir auch beauftragt, die Grundlage f\"ur die nachtr\"agliche
Verrechnung von Druckjobs der Mitarbeitenden zu schaffen.
%
Aus dieser Situation ergeben sich die folgenden {\bf Grunds\"atze}
f\"ur unser Drucksystem:
\smallskip
\item{1.} Keine zus\"atzliche Software auf den Clients.
\item{2.} Alle Druckjobs unterliegen dem Accounting.
\item{3.} Die bestehende Benutzerverwaltung verwenden.
%\item{4.} Nur authentifizierter Zugriff auf Drucker.


\section Printer Accounting

Unter Printer Accounting versteht man die Erfassung wer (wann
und wo) wieviel gedruckt hat. Auf dieser Basis k\"onnen Druckkosten
verrechnet und Statistiken erstellt werden.

Printer Accounting am PZM war zun\"achst als eine erzieherische
Massnahme gedacht, um dem verschwenderischen Umgang mit Papier
vorzubeugen.
Seit dem \"Ubergang vom PZM zur PHZ Luzern geht es zunehmend auch
um die interne \"Uberw\"alzung von Kosten zwischen Abteilungen.
Eine Forderung nach Kostenstellen f\"ur Druckjobs wird bald auf
uns zukommen.
Ob sich ein Hin- und Herschieben von Steuergeldern zwischen den
(kleinen) Abteilung ein und der selben Bildungsinstitution lohnt,
sei mal dahingestellt.

Wenn es um die Verrechnung von Druckkosten geht, dann k\"onnen
zwei Prinzipien zur Anwendung kommen, welche wir als das ``Credit''
und das ``Debit'' Prinzip bezeichnen:
\smallskip
\item{1.} {\bf Credit:} Drucken ist immer erlaubt; die Druckkosten
 werden im Nachhinein, etwa Ende Jahr oder Ende Semester, in Rechnung
 gestellt.
\item{2.} {\bf Debit:} Drucken ist nur m\"oglich, solange gen\"ugend
 ``Druckguthaben'' vorhanden ist; von diesem wird jeder Druckjob
 unmittelbar abgebucht.
\smallskip\noindent
Das Credit-Prinzip ist f\"ur Dozierende und Lehrpersonen geeignet,
das Debit-Prinzip dagegen f\"ur Studierende.


\section Das Drucksystem

Ein Drucksystem ohne Accounting ist mit freien oder kommerziellen
L\"osungen relativ einfach zu realisieren. Urspr\"unglich verwendeten
wir LPRng,\fn{LPRng: \tt www.lprng.com} eine freie Weiterentwicklung
des traditionellen BSD Print Spoolers. Vor einiger Zeit haben wir
LPRng durch CUPS\fn{CUPS: \tt www.cups.org} abgel\"ost, ein Print
Spooler welcher die freie Softwareszene im Sturm erobert.
%Implementation des \IPP,\fn{Internet Printing Protocol: \tt www.pwg.org/ipp}
%welche die freie Softwareszene im Sturm erobert.

Die Clients kommunizieren mit dem Printserver per SMB (CIFS), auch
jene, die nicht Windows als Betriebssystem verwenden. Auf dem
Printserver authentifiziert Samba\fn{Samba: \tt www.samba.org}
die Urheber von Druckjobs gegen\"uber unserem LDAP-Verzeichnis
(Grundsatz~3), nimmt Printjobs entgegen und leitet sie an den
Print Spooler (CUPS, fr\"uher LPRng) weiter.

Moderne Betriebssysteme implementieren alle das Internet Printing
Protocol (\IPP),\fn{\IPP: \tt www.pwg.org/ipp} ebenso CUPS.
Warum nicht \IPP\ verwenden? Weil Samba f\"ur uns die
Authentifizierung vornimmt. CUPS kann das zwar auch (mittels PAM an
unserem LDAP-Verzeichnis), aber die Authentifizierung zwischen Windows
und CUPS klappt per \IPP\ klappt leider nicht.\fn{Im Test mit Windows
2000 hat CUPS \"uber unbekannte \IPP\ Pakete geklagt.\\ Der Test sollte
mit aktuellen Windows-Versionen wiederholt werden.}

Alle Drucker befinden sich in einem separaten Netzwerksegment,
welches f\"ur die Clients nicht direkt erreichbar ist. Gateway
zu diesem Netzsegment ist der Printserver, der dadurch volle
Kontrolle \"uber alle Druckjobs erh\"alt (Grundsatz~2).

Druckjobs werden auf den Clients erzeugt und vom Printserver
unver\"andert zu den Druckern gesandt. Die Clients verwenden
dazu die ``normalen'' Treiber vom Hersteller, welche wir auf
dem Printserver zum Download anbieten.
Auf Windows-Clients wird der Treiber automatisch installiert
(ein Windows-Feature, von Samba unterst\"utzt).

Benutzer von Linux und Mac~OS~X brauchen eine sogenannte PPD-Datei
(PostScript Printer Description), welche einem generischen Treiber
Auskunft dar\"uber erteilt, wie der Druckjob f\"ur einen
spezifischen Drucker aufzubereiten ist.\fn{An dieser Stelle sei
noch bemerkt, dass Mac~OS~X hinter den Kulissen CUPS verwendet
und dass Apple k\"urzlich CUPS gekauft hat.}
PPD-Dateien werden vom Drucker-Hersteller zur Verf\"ugung gestellt
und k\"onnen im Internet gefunden werden.

Wir erachten die Installation von Standard-Druckertreibern auf
den Clients mit den vom Betriebssystem vorgesehenen Methoden
nicht als ``extra Software'' und somit ist auch Grundsatz~1 befolgt.


\section Accounting mit Pracc

Als wir Accounting einf\"uhrten, haben alle kommerziellen L\"osungen
die Anzahl Seiten im Druckjob gez\"ahlt, was nicht immer der Anzahl
tats\"achlich gedruckter Seiten entspricht.\fn{Papiermangel und -stau
sind die h\"aufigsten Ursachen. Gegen eine Z\"ahlung der Seiten im
Druckjob sprechen aber auch die inher\"ante Unzuverl\"assigkeit und
Sicherheits\"uberlegungen.}
% - Wer kennt schon die Formate aller Drucker/treiber?
% - Speziell pr\"aparierte Druckjobs k\"onnen:
%    -- die Z\"ahlung \"uberlisten (Zuverl\"assigkeit)
%    -- den Druckserver angreifen (Sicherheit)
%
Das war f\"ur uns inakzeptabel und bewog uns zu einer Eigenentwicklung
unter dem Namen ``Pracc.'' Zun\"achst war das ein ``Filter'' f\"ur LPRng,
sp\"ater ein ``Backend'' f\"ur CUPS.

CUPS verwendet Backends zur Kommunikation mit Druckern. Backends
kennen die spezifischen Details des Kommunikationskanals (Seriell, SMB,
JetDirect, etc). Pracc ist ein Backend f\"ur Drucker, die JetDirect
(auch bekannt als ``Port 9100'') sprechen -- das sind fast alle
Netzwerkdrucker.

Pracc pr\"uft zun\"achst, ob ein Benutzer gen\"ugend Guthaben
hat und bricht den Druckvorgang ab, falls nicht. Pracc sendet vor
und nach dem Druckjob spezielle Codes, die den Drucker veranlassen,
die Anzahl der gedruckten Seiten \"uber das Netzwerk zur\"uck an
Pracc zu melden. Das funktioniert mit unseren HP und Xerox Druckern
\"ausserst zuverl\"assig.

Problematisch sind die Kopierer, welche auch als Drucker angesprochen
werden, weil die Kommunikation unidirektional ist, also nie etwas vom
Kopierer zur\"uck kommt. F\"ur solche Ger\"ate ermitteln wir mit einer
heuristischen Methode die Anzahl Seiten im Druckjob -- mit den weiter
oben erw\"ahnten Problemen. Es ist daher wichtig, dass die Informatik-%
Abteilung bei der Beschaffung von ``Multifunktionsger\"aten'' ein
Mitspracherecht hat und ihre Empfehlungen Geh\"or finden.\fn{An der
PHZ Luzern ist die Verwaltung f\"ur die Kopierer zust\"andig.}


\section Betrieb

Die Schulleitung hat den Preis pro Seite schwarz/weiss auf 10~Rp.~und
jener f\"ur eine Seite auf einem farbigen Drucker auf 50~Rp. festgelegt.
Jedem User wird bei der ersten erfolgreichen Authentifizierung am
Printserver automatisch ein Druckkonto erstellt:
\smallskip
\item{--} Dozierende PHZ und KSM: Credit-Konto, kein initiales Guthaben
\item{--} Studierende der PHZ: Debit-Konto, Startguthaben Fr.~10
\item{--} Sch\"uler der KSM: Debit-Konto, Startguthaben Fr.~5
\smallskip\noindent
Die Startguthaben entstammen einer von den Studierenden bzw.~Sch\"ulern
bei Semesterbeginn zu entrichtenden ``Kopierpauschale.''

\"Uber eine Webseite haben alle User Einblick in ihr eigenes
(aber kein fremdes) Druckkonto mit Angaben zu Kontostand und
einer Liste aller Ausdrucke.
%
Auf der Kanzlei k\"onnen Studierende und Sch\"uler f\"ur Fr.~5
oder Fr.~10 neue Druckcredits erwerben.
Das Personal der Kanzlei nimmt einmal pro Tag die Gutschriften auf
die Druckkonti vor und f\"uhrt Buch \"uber die get\"atigten
Gutschriften (erg\"anzend zur automatischen Logf\"uhrung auf
dem Printserver).


\section Erfahrungen und Zukunft

Pracc ist seit \"uber vier Jahren in verschiedenen Versionen im
produktiven Einsatz und hat sich in dieser Zeit sehr gut bew\"ahrt.
Vor allem hat es uns erm\"oglicht, die drei Grunds\"atze f\"ur das
Drucksystem an der PHZ Luzern aufrecht zu halten.

Mit dem Wechsel von LPRng zu CUPS wurde Pracc neu geschrieben und
dabei die Z\"ahlgenauigkeit weiter verbessert. F\"ur die nahe
Zukunft sind noch die folgenden drei Erweiterungen geplant:

1.~Aufwertung: Studierende nehmen das Aufladen ihres Durckkontos selbst
vor, indem sie auf der Kanzlei einen Voucher f\"ur Fr.~5 oder Fr.~10
kaufen und den darauf vermerkten krypto\-graphi\-schen Code auf einer
Webseite eintippen (Entlastung der Kanzlei).

2.~Administration: Pracc soll vollst\"andig \"uber ein Web-Interface
administrierbar sein. Zu den administrativen Aufgaben geh\"oren: die
erw\"ahnte Aufwertung der Druckkonti Studierender, Statistiken und
Reports (f\"ur die Buchhaltung), das Stornieren von Kontobelastungen
(f\"ur administratives Personal), etc.

3.~Kostenstellen: Mitarbeitende k\"onnen nicht nur auf ihr eigenes ``Konto''
drucken, sondern auch auf Kostenstellen, f\"ur welche sie berechtigt sind.
Studierende drucken immer auf ihre privates ``Konto.''
Serverseitig ist f\"ur die Kostenstellen schon alles vorbereitet,
noch unklar ist, wie der User die Kostenstelle ausw\"ahlen kann,
ohne dass wir vom Prinzip ``keine extra Software auf den Clients''
abr\"ucken m\"ussen.


\vskip 1pc plus 1filll\nopagenumbers
\centerline{\smallrm
 Copyright 2008 by Urs-\kern-.2ex Jakob R\"uetschi and PHZ Luzern}
\bye
